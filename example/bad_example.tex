
\documentclass[13pt,a4paper]{scrartcl}
\usepackage[utf8]{inputenc}
\usepackage[english,russian]{babel}
\usepackage{indentfirst}
\usepackage{graphicx}
\usepackage{amsmath}
\usepackage{amssymb}
\usepackage{listings}
\usepackage{array}

% Некоторые множества

\def\Q{\mathbb{Q}}
\def\Z{\mathbb{Z}}
\def\N{\mathbb{N}}
\def\R{\mathbb{R}}
\def\C{\mathbb{C}}

% Бинарные операции над множествами

% xor
\def\xor{\oplus}
% объединение
\def\u{\cup}
% объединение
\def\i{\cap}


% Комбинаторика

% Биномиальный коэффициент : (из n  по k)
\def\set{\binom}
% ((из n по k))
\def\mset#1#2{\ensuremath{\left(\kern-.3em\left(\genfrac{}{}{0pt}{}{#1}{#2}\right)\kern-.3em\right)}}


% Опеации над несколькими множествами

% Сумма
\def\suml{\sum\limits}

% Сумма
\def\intl{\int\limits}

% Перемножение (знак П)
\def\prodl{\prod\limits}

% Объединение
\def\U{\bigcup}
\def\Ul#1#2#3{\U\limits_{{#1}={#2}}^{#3}}
\def\Ui#1#2{\Ul{#1}{#2}{\inf}}
\def\Uin#1#2{\U\limits_{{#1} \in {#2}}}

% Пересечение
\def\I{\bigcap}
\def\Il#1#2#3{\I\limits_{{#1}={#2}}^{#3}}
\def\Ii#1#2{\Il{#1}{#2}{\inf}}
\def\Iin#1#2{\I\limits_{{#1} \in {#2}}}


% Разделители

\def\ms{\medskip}
\def\bs{\bigskip}


% Греческий алфавит

\def\a{\alpha}
\def\b{\beta}
\def\g{\gamma}
\def\l{\lambda}
\def\e{\varepsilon}
\def\eps{\varepsilon}
\def\d{\delta}
\def\m{\mu}
\def\p{\phi}

\def\L{\Lambda}
\def\D{\Delta}
\def\M{\Mu}
\def\P{\Phi}


% Кванторы

\def\A{\forall}
\def\E{\exists\;}


% Что-то еще

\def\inf{\t{+}\infty}    % +inf
\def\O{\mathcal{O}}      %
\def\t{\text}
\def\bs{\textbackslash{}}


\begin{document}

\def\s{\sigma}
\def\a{\alpha}
\def\X{\overline{X}}
\def\Y{\overline{Y}}
\def\E{\mathbb{E}}

\section*{\text{ Домашнее }\allowbreak \text{задание }\allowbreak \text{по }\allowbreak \text{статистике }\allowbreak 03.11.17}

\subsection*{ 1.}

\(C_0 = 6,\)
\(C_1 = 7,\)
\(C_2 = 6,\)
\(C_3 = 6,\)
\(C_4 = 7,\)
\(C_5 = 6,\)
\(C_6 = 6,\)
\(C_7 = 6\)

\(k = 8; n = \suml_{i = 0}^{7} C_i = 50\)

\def\p{1 / 8}
\def\bug{C_i / 50}

\(T = 50\cdot  \suml_{i = 0}^{7} \dfrac{(\bug  - \p)^2 }{\p } = 0.24\)

\(T \sim  \chi^2(k - 1) = \chi^2(7)\)

\(pvalue = 1 - F_{\chi^2(7)}(T) \approx  0.999953 > 0.05\)
\(\Rightarrow \text{ гипотеза }\allowbreak \text{не }\allowbreak \text{отвергается }\allowbreak \text{при }\allowbreak \alpha = 0.05\)

\subsection*{ 2.}

\(C_0 = 60, C_1 = 140, C_2 = 125, C_3 = 155\)

\(k = 4; n = 480; \overline X = \frac{960 }{480}\)

\(H_0:\text{ генеральная }\allowbreak \text{совокупность }\allowbreak \xi \sim  P(\overline X) = P(2) \\\)

\(P_0 = P_{H_0}[ \xi = 0 ] = 0.135\)

\(P_1 = P_{H_0}[ \xi = 1 ] = 0.271\)

\(P_2 = P_{H_0}[ \xi = 2 ] = 0.271\)

\(P_3 = P_{H_0}[ \xi \ge  3 ] = 0.323\)

\def\bug{C_i / 480}

\(T = 480\cdot  \suml_{i = 0}^{3} \dfrac{(\bug - P_i)^2 }{P_i } = 1.31\)

\(T \sim  \chi^2(k - 2) = \chi^2(2),\text{ т.}\allowbreak \text{к. }\allowbreak \text{в }\allowbreak \text{качестве }\allowbreak \text{параметра }\allowbreak \text{использовалась}\allowbreak \)
\(\text{оценка }\allowbreak \text{максимального }\allowbreak \text{правдоподобия}\allowbreak \)

\(pvalue = 1 - F_{\chi^2(2)}(1.31) = 0.52 > 0.05\)
\(\Rightarrow \text{ гипотеза }\allowbreak \text{не }\allowbreak \text{отвергается }\allowbreak \text{при }\allowbreak \alpha = 0.05\)

\subsection*{ 3.}

\def\X{\overline{X}}
\def\Y{\overline{Y}}
\def\E{\mathbb{E}}

\(\X = 39217, S_X = 12210, n_X = n = 100\)

\(\Y = 43121, S_Y = 17020, n_Y = n = 100\)

\(H_0: \E X = \E Y\)

\(H_1: \E X \neq  \E Y\)

\(T = \dfrac{\X - \Y }{S_{X - Y} } \sim  t(df)\)

\medskip
\def\num{(S_X^2 / n_X + S_2^2 / n_Y)^2}
\def\denum{(S_X^2 / n_X)^2 / (n_X - 1) + (S_Y^2 / n_Y)^2 / (n_Y - 1)}
\(df = \dfrac{\num }{\denum }\)

\(\text{Поскольку }\allowbreak n_X = n_Y = n, df = \dfrac{(S_X^2 + S_Y^2)^2 (n - 1) }{S_X^4 + S_Y^2 }\)
\(= \dfrac{99\cdot  (12210^2 + 17020^2)^2 }{12210^4 + 17020^4 } = 179.56\)
\medskip

\(S_{X - Y} = \sqrt{ \dfrac{S_X^2 }{n_X } + \dfrac{S_Y^2 }{n_Y } }\)
\(= \sqrt{ \dfrac{12210^2 }{100 } + \dfrac{17020^2 }{100 } }\)
\(= 2094\)
\medskip

\(T = \dfrac{39217 - 43121 }{2094 } = -1.864\)

\medskip
\(T\text{ попадает }\allowbreak \text{в }\allowbreak 99\%\text{-процентный }\allowbreak \text{доверительный }\allowbreak \text{интервал }\allowbreak = (-2.6; 2.6)\)

\(\text{Таким }\allowbreak \text{образом, }\allowbreak \text{гипотеза }\allowbreak \text{не }\allowbreak \text{отвергается}\allowbreak \)

\subsection*{ 4\text{ (метод }\allowbreak \text{генерации }\allowbreak \text{нормальных }\allowbreak \text{распределений }\allowbreak \text{с }\allowbreak \text{заданной }\allowbreak \text{корреляцией)}\allowbreak }

\def\tmp{\sqrt{1 - \rho^2}}

\(X \sim  N(0, 1),\)
\(Z \sim  N(0, 1),\)
\(Y = \rho X + \tmp Z\)

\(r(X, Y)\)
\(= \frac{cov(X, Y) }{\sigma X \sigma Y }\)
\(= cov(X, Y)\)
\(= cov(X, \rho X) + cov(X, \tmp Z) \\\)
\(= \rho\cdot  cov(X, X)\)
\(= \rho\)

\end{document}
